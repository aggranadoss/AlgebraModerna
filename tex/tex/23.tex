\section*{Ejercicios 23}

\textbf{División de Polinomios en Z\textsubscript{p}[x]}
	
\begin{enumerate}
	\item Dados \(f(x) = x^6 + 3x^5 + Ax^2 - 3x + 2\) y \(g(x) = x^2 + 2x - 3\) en \(\mathbb{Z}_7[x]\), encuentra \(q(x)\) y \(r(x)\) según el algoritmo de división, de manera que \(f(x) = g(x)q(x) + r(x)\), con \(r(x) = 0\) o de grado menor que el de \(g(x)\).
		
	\item Dados \(f(x) = -x^3 + 3x^5 + 4x^2 - 3x + 2\) y \(g(x) = 3x^2 + 2x - 3\) en \(\mathbb{Z}_7[x]\), encuentra \(q(x)\) y \(r(x)\) según el algoritmo de división, de manera que \(f(x) = g(x)q(x) + r(x)\), con \(r(x) = 0\) o de grado menor que el de \(g(x)\).
		
	\item Dados \(f(x) = x^5 - 2x^4 + 3x - 5\) y \(g(x) = 2x + 1\) en \(\mathbb{Z}[x]\), encuentra \(q(x)\) y \(r(x)\) según el algoritmo de división, de manera que \(f(x) = g(x)q(x) + r(x)\), con \(r(x) = 0\) o de grado menor que el de \(g(x)\).
		
	\item Dados \(f(x) = x^4 + 5x^3 - 3x^2\) y \(g(x) = 5x^2 - x + 2\) en \(\mathbb{Z}[x]\), encuentra \(q(x)\) y \(r(x)\) según el algoritmo de división, de manera que \(f(x) = g(x)q(x) + r(x)\), con \(r(x) = 0\) o de grado menor que el de \(g(x)\).
\end{enumerate}
	
\textbf{Grupos Multiplicativos Cíclicos de Campos Finitos}
	
\begin{enumerate}
	\setcounter{enumi}{4}
	\item Encuentra todos los generadores del grupo multiplicativo cíclico de unidades del campo finito \(\mathbb{Z}_5\).
		
	\item Encuentra todos los generadores del grupo multiplicativo cíclico de unidades del campo finito \(\mathbb{Z}_7\).
		
	\item Encuentra todos los generadores del grupo multiplicativo cíclico de unidades del campo finito \(\mathbb{Z}_{17}\).
		
	\item Encuentra todos los generadores del grupo multiplicativo cíclico de unidades del campo finito \(\mathbb{Z}_{23}\).
\end{enumerate}
	
\textbf{Factorización de Polinomios en \(\mathbb{Z}[x]\)}
	
\begin{enumerate}
	\setcounter{enumi}{8}
	\item El polinomio \(x^4 + 4\) se puede factorizar en factores lineales en \(\mathbb{Z}[x]\). Encuentra esta factorización.
		
	\item El polinomio \(x^3 + 2x^2 + 2x + 1\) se puede factorizar en factores lineales en \(\mathbb{Z}_7[x]\). Encuentra esta factorización.
		
	\item El polinomio \(2x^3 + 3x^2 - x - 5\) se puede factorizar en factores lineales en \(\mathbb{Z}_n[x]\). Encuentra esta factorización.
		
	\item ¿Es \(x^3 + 2x + 3\) un polinomio irreducible en \(\mathbb{Z}_5[x]\)? ¿Por qué? Exprésalo como un producto de polinomios irreducibles en \(\mathbb{Z}_5[x]\).
		
	\item ¿Es \(2x^3 + x^2 + 2x + 2\) un polinomio irreducible en \(\mathbb{Z}_5[x]\)? ¿Por qué? Exprésalo como un producto de polinomios irreducibles en \(\mathbb{Z}_5[x]\).
		
	\item Demuestra que \(f(x) = x^2 + 8x - 2\) es irreducible sobre \(\mathbb{Q}\). ¿Es irreducible sobre \(\mathbb{E}\)? ¿Sobre \(\mathbb{C}\)?
		
	\item Repite el Ejercicio 14 con \(g(x) = x^2 + 6x + 12\) en lugar de \(f(x)\).
		
	\item Demuestra que \(x^3 + 3x^2 - 8\) es irreducible sobre \(\mathbb{Q}\).
		
	\item Demuestra que \(x^4 - 22x^2 + 1\) es irreducible sobre \(\mathbb{Q}\).
\end{enumerate}
	
\textbf{Criterio de Eisenstein}
	
\begin{enumerate}
	\setcounter{enumi}{17}
	\item Determina si el polinomio \(x^2 - 12\) satisface el criterio de Eisenstein para irreducibilidad sobre \(\mathbb{Q}\).
		
	\item Determina si el polinomio \(8x^3 + 6x^2 - 9x + 2A\) satisface el criterio de Eisenstein para irreducibilidad sobre \(\mathbb{Q}\).
		
	\item Determina si el polinomio \(4x^{10} - 9x^3 + 2Ax - 18\) satisface el criterio de Eisenstein para irreducibilidad sobre \(\mathbb{Q}\).
		
	\item Determina si el polinomio \(2x^{10} - 25x^3 + 10x^2 - 30\) satisface el criterio de Eisenstein para irreducibilidad sobre \(\mathbb{Q}\).
\end{enumerate}

\textbf{Encontrar las Raíces de un Polinomio}
	
\begin{enumerate}
	\setcounter{enumi}{21}
	\item Encuentra todas las raíces de \(6x^4 + 17x^3 + 11x^2 + x - 10\) en \(\mathbb{Q}\).
\end{enumerate}

\subsection*{Teoría}

\begin{enumerate}
	\item Demuestra que para \(p\) un número primo, el polinomio \(x^p + a\) en \(\mathbb{Z}_p[x]\) no es irreducible para ningún \(a \in \mathbb{Z}_p\).
	
	\item Si \(F\) es un campo y \(a^0\) es una raíz de \(f(x) = a_0 + a_1x + a_2x^2 + \ldots + a_nx^n\) en \(F[x]\), demuestra que \(1/a\) es una raíz de \(a_n + a_{n-1}x + \ldots + a_0x^n\).
	
	\item (Teorema del Resto) Sea \(f(x) \in F[x]\), donde \(F\) es un campo, y sea \(a \in F\). Demuestra que el resto \(r(x)\) cuando \(f(x)\) se divide por \(x - a\), de acuerdo con el algoritmo de división, es \(f(a)\).
	
	\item Sea \(\phi_m: \mathbb{Z}[x] \rightarrow \mathbb{Z}_m[x]\) dada por
	\[
	\phi_m(a_0 + a_1x + a_2x^2 + \ldots + a_nx^n) = \phi_m(a_0) + \phi_m(a_1)x + \phi_m(a_2)x^2 + \ldots + \phi_m(a_n)x^n,
	\]
	donde \(\phi_m\) es la aplicación natural \(\mod m\) definida por \(\phi_m(a) =\) (el resto de \(a\) al dividirlo por \(m\)) para \(a \in \mathbb{Z}\).
	
	a. Demuestra que \(\phi_m\) es un homomorfismo de \(\mathbb{Z}[x]\) a \(\mathbb{Z}_m[x]\).
	
	b. Demuestra que si \(f(x) \in \mathbb{Z}[x]\) y \(\phi_m(f(x))\) tienen ambas grado \(n\) y \(a \cdot \phi_m(f(x))\) no se factoriza en \(\mathbb{Z}_m[x]\) en dos polinomios de grado menor que \(n\), entonces \(f(x)\) es irreducible en \(\mathbb{Q}[x]\).
	
	c. Usa la parte (b) para demostrar que \(x^3 + 11x + 36\) es irreducible en \(\mathbb{Q}[x]\). [Pista: Prueba con un valor primo de \(m\) que simplifique los coeficientes.]
\end{enumerate}


\subsection*{Soluciones}


\textbf{Generadores de Grupos Multiplicativos Cíclicos en Campos Finitos}
	
\begin{enumerate}
	\item Para \(2 \in \mathbb{Z}_5\), tenemos que \(2^2 = 4\), \(2^3 = 3\), \(2^4 = 1\), por lo que 2 genera el subgrupo multiplicativo \(\{1, 2, 3, 4\}\) de todas las unidades en \(\mathbb{Z}_5\). Según el Corolario 6.16, los únicos generadores son \(2^1 = 2\) y \(2^3 = 3\).
		
	\item Para \(2 \in \mathbb{Z}_7\), encontramos que \(2^3 = 1\), por lo que 2 no genera. Probando con \(3\), encontramos que \(3^2 = 2\), \(3^3 = 6\), \(3^4 = 4\), \(3^5 = 5\), y \(3^6 = 1\), por lo que 3 genera las seis unidades \(1, 2, 3, 4, 5, 6\) en \(\mathbb{Z}_7\). Por el Corolario 6.16, los únicos generadores son \(3^1 = 3\) y \(3^5 = 5\).
		
	\item Para \(2 \in \mathbb{Z}_{17}\), encontramos que \(2^4 = -1\), por lo que \(2^8 = 1\) y 2 no genera. Probando con \(3\), encontramos que \(3^2 = 9\), \(3^3 = 10\), \(3^4 = 13\), \(3^5 = 5\), \(3^6 = 15\), \(3^7 = 11\), \(3^8 = 16 = -1\). Dado que el orden de 3 debe dividir 16, vemos que 3 debe tener orden 16, por lo que 3 genera las unidades en \(\mathbb{Z}_{17}\). Por el Corolario 6.16, los únicos generadores son \(3^1 = 3\), \(3^3 = 10\), \(3^5 = 5\), \(3^7 = 11\), \(3^9 = 14\), \(3^{11} = 7\), \(3^{13} = 12\), y \(3^{15} = 6\).
		
	\item Para \(2 \in \mathbb{Z}_{23}\), encontramos que \(2^2 = 4\), \(2^3 = 8\), \(2^4 = 16\), \(2^5 = 9\), \(2^6 = 18\), \(2^7 = 13\), \(2^8 = 3\), \(2^9 = 6\), \(2^{10} = 12\), y \(2^{11} = 1\), por lo que 2 no genera. Sin embargo, esta computación muestra que \((-2)^{11} = -1\). Dado que el orden de \(-2\) debe dividir 22, vemos que \((-2)^1 = 2\) debe tener orden 22, por lo que \((-2)^1\) genera las unidades en \(\mathbb{Z}_{23}\). Por el Corolario 6.16, los únicos generadores son \((-2)^1 = 2\), \((-2)^3 = 15\), \((-2)^5 = 14\), \((-2)^7 = 10\), \((-2)^9 = 17\), \((-2)^{13} = 19\), \((-2)^{15} = 7\), \((-2)^{17} = 5\), \((-2)^{19} = 20\), y \((-2)^{21} = 11\).
	\end{enumerate}
	
\textbf{Factorización de Polinomios en \(\mathbb{Z}[x]\)}
	
\begin{enumerate}
	\item En \(\mathbb{Z}_5\), tenemos \(x^4 + 4 = x^4 - 1 = (x^2 + 1)(x^2 - 1)\). Reemplazando 1 por -4 nuevamente, continuamos y descubrimos que \((x^2 - 4)(x^2 - 1) = (x - 2)(x + 2)(x - 1)(x + 1)\).
		
	\item Por inspección, -1 es una raíz de \(x^3 + 2x^2 + 2x + 1\) en \(\mathbb{Z}_7[x]\). Ejecutando el algoritmo de división como se ilustra en nuestras respuestas a los Ejercicios 1 a 3, calculamos \(x^3+2x^2+2x+1\) dividido por \(x - (-1) = x + 1\), y encontramos que \(x^3+2x^2+2x+1 = (x+1)(x^2+x+1)\). Por inspección, 2 y 4 son raíces de \(x^2 + x + 1\). Así que la factorización es \(x^3+2x^2+2x+1 = (x+1)(x-4)(x-2)\).
		
	\item Por inspección, 3 es una raíz de \(2x^3 + 3x^2 - 7x - 5\) en \(\mathbb{Z}_{11}[x]\). Dividiendo por \(x - 3\) usando la técnica ilustrada en nuestras respuestas a los Ejercicios 1 a 3, encontramos que \(2x^3+3x^2-7x-5 = (x-3)(2)(x^2-x-1)\). Por inspección, -3 y 4 son raíces de \(x^2 - x - 1\), por lo que la factorización es \(2x^3+3x^2-7x-5 = (x-3)(x+3)(2x-8)\).
		
	\item Por inspección, -1 es una raíz de \(x^3 + 2x + 3\) en \(\mathbb{Z}_5[x]\), por lo que el polinomio no es irreducible. Dividiendo por \(x + 1\) usando la técnica de los Ejercicios 1 a 3, obtenemos \(x^3+2x+3 = (x+1)(x^2-x+3)\). Por inspección, -1 y 2 son raíces de \(x^2 - x + 3\), por lo que la factorización es \(x^3+2x+3=(x+1)(x+1)(x-2)\).
\end{enumerate}
	
\textbf{Irreducibilidad y Factorización de Polinomios}
	
\begin{enumerate}
	\item Sea \(f(x) = 2x^3 + x^2 + 2x + 2\) en \(\mathbb{Z}_5[x]\). Entonces \(f(0) = 2\), \(f(1) = 2\), \(f(-1) = -1\), \(f(2) = 1\), y \(f(-2) = 1\), por lo que \(f(x)\) no tiene ceros en \(\mathbb{Z}_5\). Dado que \(f(x)\) es de grado 3, el Teorema 23.10 muestra que \(f(x)\) es irreducible sobre \(\mathbb{Z}_5\).
		
	\item \(f(x) = x^2 + 8x - 2\) satisface la condición de Eisenstein para irreducibilidad sobre \(\mathbb{Q}\) con \(p = 2\). No es irreducible sobre \(\mathbb{R}\) porque la fórmula cuadrática muestra que tiene raíces reales \((-8 \pm \sqrt{72})/2\). Por supuesto, tampoco es irreducible sobre \(\mathbb{C}\).
		
	\item El polinomio \(g(x) = x^2 + 6x + 12\) es irreducible sobre \(\mathbb{Q}\) porque satisface la condición de Eisenstein con \(p = 3\). También es irreducible sobre \(\mathbb{R}\) porque la fórmula cuadrática muestra que sus raíces son \((-6 \pm \sqrt{-12})/2\), que no están en \(\mathbb{R}\). No es irreducible sobre \(\mathbb{C}\) porque sus raíces están en \(\mathbb{C}\).
		
	\item Si \(x^3 + 3x^2 - 8\) es reducible sobre \(\mathbb{Q}\), entonces, por el Teorema 23.11, se factoriza en \(Z[x]\) y debe tener un factor lineal de la forma \(x - a\) en \(Z[x]\). Entonces, \(a\) debe ser una raíz del polinomio y debe dividir a -8, por lo que las posibilidades son \(a = \pm 1, \pm 2, \pm 4, \pm 8\). Calculando el polinomio en estos valores, encontramos que ninguno de ellos es raíz del polinomio, que es entonces irreducible sobre \(\mathbb{Q}\).
		
	\item Si \(x^4 - 22x^2 + 1\) es reducible sobre \(\mathbb{Q}\), entonces, por el Teorema 23.11, se factoriza en \(Z[x]\) y debe ser un factor lineal en \(Z[x]\) o factorizar en dos cuadráticos en \(Z[x]\). Las únicas posibilidades para un factor lineal son \(x \pm 1\), y claramente ni 1 ni -1 son raíces del polinomio, por lo que un factor lineal es imposible. Supongamos
		\[
		x^4 - 22x^2 + 1 = (x^2 + ax + b)(x^2 + cx + d).
		\]
		Igualando coeficientes, vemos que el coeficiente \(x^3\) es 0, por lo que \(a + c = 0\), el coeficiente \(x^2\) es -22, por lo que \(ac + b + d = -22\), el coeficiente \(x\) es 0, por lo que \(bc + ad = 0\), y el término constante es 1, por lo que \(bd = 1\). Entonces, \(b = d = 1\) o \(b = d = -1\).
		
		Supongamos \(b = d = 1\). Entonces, \(-22 = ac + 1 + 1\), así que \(ac = -24\). Debido a que \(a + c = 0\), tenemos \(a = -c\), por lo que \(-c^2 = -24\), lo cual es imposible para un entero \(c\). Similarmente, si \(b = d = -1\), deducimos que \(-c^2 = -20\), lo cual también es imposible. Por lo tanto, el polinomio es irreducible.
\end{enumerate}
	
\textbf{Criterio de Eisenstein}
	
\begin{enumerate}
	\item Sí, con \(p = 3\).
		
	\item Sí, con \(p = 3\).
		
	\item No, ya que 2 divide al coeficiente 4 de \(4x^{10} - 9x^3 + 2Ax - 18\) y 32 divide al término constante -18.
		
	\item Sí, con \(p = 5\).
\end{enumerate}
	
\textbf{Encontrar las Raíces de un Polinomio}
	
\begin{enumerate}
	\item Encuentra todas las raíces de \(6x^4 + 17x^3 + 11x^2 + x - 10\) en \(\mathbb{Q}\).
\end{enumerate}
\begin{enumerate}
	\item Observa que \(x^2 = x \cdot x\) y \(x^2 + 1 = (x + 1)^2\) son reducibles en \(\mathbb{Z}_p\). Para un número primo impar \(p\) y \(a \in \mathbb{Z}_p\), sabemos que \((-a)^p + a = -a^p + a = -a + a = 0\) por el Corolario 20.2. Por lo tanto, \(x^p + a\) tiene a \(-a\) como raíz, por lo que es reducible sobre \(\mathbb{Z}_p\) para todo primo \(p\). [De hecho, el teorema binómico y el Corolario 20.2 muestran que \(x^p + a = (x + a)^p\). 
	
	\item Dado que \(f(a) = a_0 + a_1a + \ldots + a_na^n = 0\) y \(a^n \neq 0\), al dividir por \(a^n\), obtenemos \(a_0 \left( \frac{1}{a} \right)^n + a_1 \left( \frac{1}{a} \right)^{n-1} + \ldots + a_n = 0\), que es lo que queríamos mostrar.
	
	\item Por el Teorema 23.1, sabemos que \(f(x) = q(x)(x - a) + c\) para alguna constante \(c \in F\). Aplicando el homomorfismo de evaluación \(\phi_a\) a ambos lados de esta ecuación, obtenemos \(f(a) = q(a)(a - a) + c = q(a) \cdot 0 + c = c\), por lo que el resto \(r(x) = c\) es realmente \(f(a)\).
	
	\item 
	
	a. Sea \(f(x) = \sum_{i=0}^{\infty} a_i x^i\) y \(g(x) = \sum_{i=0}^{\infty} b_i x^i\). Entonces,
	\[
	\sigma_m(f(x) + g(x)) = \sum_{i=0}^{\infty} (\sigma_m(a_i) + \sigma_m(b_i))x^i = \sigma_m(f(x)) + \sigma_m(g(x)),
	\]
	y
	\[
	\sigma_m(f(x) \cdot g(x)) = \sum_{n=0}^{\infty} \left( \sum_{i=0}^{n} \sigma_m(a_i b_{n-i}) \right) x^n = \sigma_m(f(x)) \cdot \sigma_m(g(x)),
	\]
	por lo que \(\sigma_m\) es un homomorfismo. Si \(h(x) \in \mathbb{Z}_m[x]\), entonces si \(k(x)\) es el polinomio en \(\mathbb{Z}[x]\) obtenido de \(h(x)\) al considerar solo los coeficientes como elementos de \(\mathbb{Z}\) en lugar de \(\mathbb{Z}_m\), vemos que \(\sigma_m(k(x)) = h(x)\), por lo que el homomorfismo \(\sigma_m\) es sobre \(\mathbb{Z}_m[x]\).
	
	b. Supongamos que \(f(x) = g(x)h(x)\) para \(g(x)\), \(h(x) \in \mathbb{Z}[x]\) con los grados tanto de \(g(x)\) como de \(h(x)\) menores que el grado \(n\) de \(f(x)\). Aplicando el homomorfismo \(\sigma_m\), vemos que \(\sigma_m(f(x)) = \sigma_m(g(x)) \cdot \sigma_m(h(x))\) es una factorización de \(\sigma_m(f(x))\) en dos polinomios de grado menor que el grado \(n\) de \(\sigma_m(f(x))\), lo cual es contrario a la hipótesis. Por lo tanto, \(f(x)\) es irreducible en \(\mathbb{Q}[x]\).
	
	c. Tomando \(m = 5\), vemos que \(\sigma_5(x^3 + 17x + 36) = x^3 + 2x + 1\), el cual no tiene ninguno de los cinco elementos 0, 1, -1, 2, -2 de \(\mathbb{Z}_5\) como cero, y por lo tanto, es irreducible sobre \(\mathbb{Z}_5\) por el Teorema 23.10. Por la Parte (b), concluimos que \(x^3 + 17x + 36\) es irreducible sobre \(\mathbb{Q}\).
	
\end{enumerate}
