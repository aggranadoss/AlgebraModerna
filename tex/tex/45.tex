
\subsection*{Sección 45}

\noindent En los Ejercicios 1 a 8, determine si el elemento es irreducible en el dominio indicado.
\begin{enumerate}
    \item $5$ en $\mathbb{Z}$
    
    \textbf{Solución:} Sí, $5$ es irreducible en $\mathbb{Z}$.

    \item $-17$ en $\mathbb{Z}$
    
    \textbf{Solución:} Sí, $-17$ es irreducible en $\mathbb{Z}$.

    \item $14$ en $\mathbb{Z}$
    
    \textbf{Solución:} No, $14 = 2 \cdot 7$ no es irreducible en $\mathbb{Z}$.

    \item $2x - 3$ en $\mathbb{Z}[x]$
    
    \textbf{Solución:} Sí, $2x - 3$ es irreducible en $\mathbb{Z}[x]$.

    \item $2x - 10$ en $\mathbb{Z}[x]$
    
    \textbf{Solución:} No, $2x - 10 = 2(x - 5)$ no es irreducible en $\mathbb{Z}[x]$.

    \item $2x - 3$ en $\mathbb{Q}[x]$
    
    \textbf{Solución:} Sí, $2x - 3$ es irreducible en $\mathbb{Q}[x]$.

    \item $2x - 10$ en $\mathbb{Q}[x]$
    
    \textbf{Solución:} Sí, $2x - 10$ es irreducible en $\mathbb{Q}[x]$, ya que $2$ es una unidad en este dominio.

    \item $2x - 10$ en $\mathbb{Z}_{11}[x]$
    
    \textbf{Solución:} Sí, $2x - 10$ es irreducible en $\mathbb{Z}_{11}[x]$, ya que $2$ es una unidad en este dominio.
\end{enumerate}
    

\begin{enumerate}
    \setcounter{enumi}{8}
    \item Si es posible, dé cuatro diferentes asociados de $2x - 7$ visto como un elemento de $\mathbb{Z}[x]$; de $\mathbb{Q}[x]$; de $\mathbb{Z}_{11}[x]$.
    
    \textbf{Solución:} (Ver la respuesta en el texto.)

    \item Factorice el polinomio $4x^2 - 4x + 8$ en un producto de irreducibles viéndolo como un elemento del dominio integral $\mathbb{Z}[x]$; del dominio integral $\mathbb{Q}[x]$; del dominio integral $\mathbb{Z}_{11}[x]$.
    
    \textbf{Solución:} En $\mathbb{Z}[x]$, $4x^2 - 4x + 8 = (2)(2)(x^2 - x + 2)$. El polinomio cuadrático es irreducible porque sus ceros son números complejos.\\
    En $\mathbb{Q}[x]$, $4x^2 - 4x + 8$ ya es irreducible porque $4$ es una unidad y los ceros del polinomio son números complejos.\\
    En $\mathbb{Z}_{11}[x]$, $4x^2 - 4x + 8 = (4x + 2)(x + 4)$. Encontramos la factorización descubriendo que $-4$ y $5$ son ceros del polinomio. Nótese que $2$ es una unidad.

    \item En los Ejercicios 11 a 13, encuentre todos los mcd de los elementos dados de $\mathbb{Z}$.
    \begin{enumerate}
        \item $234$, $3250$, $1690$
        
        \textbf{Solución:} Procedemos factorizando el número más pequeño en irreducibles y, usando una calculadora, descubrimos cuáles irreducibles dividen a los números más grandes. Encontramos que $234 = 2 \cdot 117 = 2 \cdot 9 \cdot 13$. Nuestra calculadora muestra que $9$ no divide $3250$, pero $2$ y $13$ sí, y ambos $2$ y $13$ dividen $1690$. Así, los mcd son $26$ y $-26$.

        \item $784$, $-1960$, $448$
        
        \textbf{Solución:} Procedemos factorizando el número más pequeño en irreducibles y, usando una calculadora, descubrimos cuáles irreducibles dividen a los números más grandes. Encontramos que $448 = 4 \cdot 112 = 4 \cdot 4 \cdot 28 = 2^6 \cdot 7$. Nuestra calculadora muestra que $7$ divide tanto $784$ como $1960$, y que la mayor potencia de $2$ que divide $784$ es $16$ mientras que la mayor potencia que divide $1960$ es $8$. Así, los mcd son $8 \cdot 7 = 56$ y $-56$.

        \item $2178$, $396$, $792$, $594$
        
        \textbf{Solución:} Procedemos factorizando el número más pequeño en irreducibles y, usando una calculadora, descubrimos cuáles irreducibles dividen a los números más grandes. Encontramos que $396 = 6 \cdot 66 = 6 \cdot 6 \cdot 11 = 2^2 \cdot 3^2 \cdot 11$. Nuestra calculadora muestra que tanto $11$ como $9$ dividen a los otros tres números, pero $2178$ y $594$ no son divisibles por $4$, pero sí son divisibles por $2$. Así, los mcd son $11 \cdot 9 \cdot 2 = 198$ y $-198$.
    \end{enumerate}

    \item En los Ejercicios 14 a 17, exprese el polinomio dado como el producto de su contenido con un polinomio primitivo en el DFI indicado.
    \begin{enumerate}
        \item $18x^2 - 12x + 48$ en $\mathbb{Z}[x]$
        
        \textbf{Solución:} $18x^2 - 12x + 48 = 6(3x^2 - 2x + 8)$.

        \item $18x^2 - 12x + 48$ en $\mathbb{Q}[x]$
        
        \textbf{Solución:} Como cada $q \in \mathbb{Q}$ no nulo es una unidad en $\mathbb{Q}[x]$, podemos "factorizar" cualquier constante racional no nula como el contenido (unidad) de este polinomio. Por ejemplo,\\
        $(1)(18x^2 - 12x + 48)$ y $\frac{1}{2}(36x^2 - 24x + 96)$ son dos de un número infinito de posibles respuestas.

        \item $2x^2 - 3x + 6$ en $\mathbb{Z}[x]$
        
        \textbf{Solución:} La factorización es $(1)(2x^2 - 3x + 6)$ porque el polinomio es primitivo.

        \item $2x^2 - 3x + 6$ en $\mathbb{Z}_7[x]$
        
        \textbf{Solución:} Como cada $a \in \mathbb{Z}_7$ no nulo es una unidad en $\mathbb{Z}_7[x]$, podemos "factorizar" cualquier constante no nula como el contenido (unidad) de este polinomio. Por ejemplo,\\
        $(1)(2x^2 - 3x + 6)$ y $(5)(6x^2 + 5x + 4)$ son dos de un número infinito de posibles respuestas.
    \end{enumerate}
\end{enumerate}
     
\begin{enumerate}
    \setcounter{enumi}{24}
    \item Demuestre que si \( p \) es un primo en un dominio integral \( D \), entonces \( p \) es irreducible.
    
    \textbf{Solución:} Sea \( p \) un primo de \( D \), y supongamos que \( p = ab \) para algunos \( a, b \in D \). Entonces \( ab = (1)p \), por lo que \( p \) divide a \( ab \) y, por lo tanto, divide a \( a \) o \( b \), porque \( p \) es un primo. Supongamos que \( a = pc \). Entonces \( p = (1)p = pcb \) y la cancelación en el dominio integral produce \( 1 = cb \), por lo que \( b \) es una unidad de \( D \). Del mismo modo, si \( p \) divide \( b \), concluimos que \( a \) es una unidad en \( D \). Por lo tanto, \( a \) o \( b \) es una unidad, así que \( p \) es irreducible.
    
    \item Demuestre que si \( p \) es un irreducible en un DFI, entonces \( p \) es un primo.
    
    \textbf{Solución:} Sea \( p \) un irreducible en un DFI, y supongamos que \( p \) divide a \( ab \). Debemos demostrar que \( p \) divide a \( a \) o \( p \) divide a \( b \). Sea \( ab = pc \), y factorizamos \( ab \) en irreducibles factorizando primero \( a \) en irreducibles, luego factorizando \( b \) en irreducibles, y finalmente tomando el producto de estas dos factorizaciones. Ahora, \( ab \) también podría factorizarse en irreducibles tomando \( p \) y una factorización de \( c \) en irreducibles. Dado que la factorización en irreducibles en un DFI es única hasta el orden y los asociados, debe ser que un asociado de \( p \) aparece en la primera factorización, formada por los factores de \( a \) y los factores de \( b \). Así, un asociado de \( p \), digamos \( up \), aparece en la factorización de \( a \) o en la factorización de \( b \). Se deduce de inmediato que \( p \) divide a \( a \) o \( b \).

    \item Para un anillo conmutativo \( R \) con unidad, muestre que la relación \( a \sim b \) si \( a \) es un asociado de \( b \) (es decir, si \( a = bu \) para una unidad \( u \) en \( R \)) es una relación de equivalencia en \( R \).
    
    \textbf{Solución:}
    \begin{itemize}
        \item Reflexiva: \( a = a \cdot 1 \), por lo que \( a \sim a \).
        \item Simétrica: Supongamos \( a \sim b \), de modo que \( a = bu \) para una unidad \( u \). Entonces \( u^{-1} \) es una unidad y \( b = au^{-1} \), por lo que \( b \sim a \).
        \item Transitiva: Supongamos que \( a \sim b \) y \( b \sim c \). Entonces hay unidades \( u_1 \) y \( u_2 \) tales que \( a = bu_1 \) y \( b = cu_2 \). Sustituyendo, tenemos \( a = cu_2 u_1 = c(u_2 u_1) \). Como el producto \( u_2 u_1 \) de dos unidades es de nuevo una unidad, encontramos que \( a \sim c \).
    \end{itemize}

    \item Sea \( D \) un dominio integral. El Ejercicio 37, Sección 18 mostró que \( \langle U, \cdot \rangle \) es un grupo donde \( U \) es el conjunto de unidades de \( D \). Muestre que el conjunto \( D^* - U \) de los no unidades de \( D \) excluyendo el 0 está cerrado bajo la multiplicación. ¿Es este conjunto un grupo bajo la multiplicación de \( D \)?
    
    \textbf{Solución:} Sea \( a \) y \( b \) no unidades en \( D^* - U \). Supongamos que \( ab \) es una unidad, de modo que \( (ab)c = 1 \) para algún \( c \in D \). Entonces \( a(bc) = 1 \) y \( a \) es una unidad, lo cual es contrario a nuestra elección de \( a \). Por lo tanto, \( ab \) es de nuevo una no unidad, y \( ab \neq 0 \) porque \( D \) no tiene divisores de cero. Así, \( ab \in D^* - U \). Vemos que \( D^* - U \) no es un grupo, porque la identidad multiplicativa es una unidad y, por lo tanto, no está en \( D^* - U \).

    \item Sea \( D \) un DFI. Muestre que un divisor no constante de un polinomio primitivo en \( D[x] \) es nuevamente un polinomio primitivo.
    
    \textbf{Solución:} Sea \( g(x) \) un divisor no constante del polinomio primitivo \( f(x) \) en \( D[x] \). Supongamos que \( f(x) = g(x)q(x) \). Como \( D \) es un DFI, sabemos que \( D[x] \) también es un DFI. Factorizamos \( f(x) \) en irreducibles factorizando cada uno de \( g(x) \) y \( q(x) \) en irreducibles, y luego tomando el producto de estas factorizaciones. Cada factor no constante que aparece es un irreducible en \( D[x] \) y, por lo tanto, es un polinomio primitivo. Como el producto de polinomios primitivos es primitivo por el Corolario 45.26, vemos que el contenido de \( g(x)q(x) \) es el producto del contenido de \( g(x) \) y el contenido de \( q(x) \), y debe ser el mismo (hasta un factor unidad) que el contenido de \( f(x) \). Pero \( f(x) \) tiene contenido 1 porque es primitivo. Así, \( g(x) \) y \( q(x) \) tienen contenido 1. Por lo tanto, \( g(x) \) es un producto de polinomios primitivos, así que es primitivo por el Corolario 45.26.

    \item Sea \( N \) un ideal en un PID \( D \). Si \( N \) no es maximal, entonces hay un ideal propio \( N_1 \) de \( D \) tal que \( N \subset N_1 \). Si \( N_1 \) no es maximal, encontramos un ideal propio \( N_2 \) tal que \( N_1 \subset N_2 \). Continuando este proceso, construimos una cadena \( N \subset N_1 \subset N_2 \subset \cdots \subset N_i \) de ideales propios, cada uno propiamente contenido en el siguiente excepto por el último ideal. Como un PID satisface la condición de la cadena ascendente, no podemos extender esto a una cadena infinita, por lo que después de un número finito de pasos debemos encontrar un ideal propio \( N_r \) que contenga a \( N \) y que no esté propiamente contenido en ningún ideal propio de \( D \). Es decir, alcanzamos un ideal maximal \( N_r \) de \( D \) que contiene a \( N \).

    \item Factorice \( x^3 - y^3 \) en irreducibles en \( \mathbb{Q}[x, y] \) y demuestre que cada uno de los factores es irreducible.
    
    \textbf{Solución:} Tenemos \( x^3 - y^3 = (x - y)(x^2 + xy + y^2) \). Por supuesto, \( x - y \) es irreducible. Afirmamos que \( x^2 + xy + y^2 \) es irreducible en \( \mathbb{Q}[x, y] \). Supongamos que \( x^2 + xy + y^2 \) se factoriza en un producto de dos polinomios que no son unidades en \( \mathbb{Q}[x, y] \). Tal factorización tendría que ser de la forma \( x^2 + xy + y^2 = (ax + by)(cx + dy) \) con \( a, b, c \) y \( d \) todos elementos no nulos de \( \mathbb{Q} \). Consideremos el homomorfismo de evaluación \( \phi_1 : (\mathbb{Q}[x])[y] \to \mathbb{Q}[x] \) tal que \( \phi(y) = 1 \). Aplicando \( \phi_1 \) a ambos lados de dicha factorización obtendríamos \( x^2 + x + 1 = (ax + b)(cx + d) \). Pero \( x^2 + x + 1 \) es irreducible en \( \mathbb{Q}[x] \) porque sus ceros son complejos, por lo que no existe tal factorización. Esto muestra que \( x^2 + xy + y^2 \) es irreducible en \( (\mathbb{Q}[x])[y] \), que es isomórfico a \( \mathbb{Q}[x, y] \) bajo un isomorfismo que identifica \( y^2 + yx + x^2 \) y \( x^2 + xy + y^2 \).

\end{enumerate}

